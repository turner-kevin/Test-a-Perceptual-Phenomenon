
% Default to the notebook output style

    


% Inherit from the specified cell style.




    
\documentclass[11pt]{article}

    
    
    \usepackage[T1]{fontenc}
    % Nicer default font (+ math font) than Computer Modern for most use cases
    \usepackage{mathpazo}

    % Basic figure setup, for now with no caption control since it's done
    % automatically by Pandoc (which extracts ![](path) syntax from Markdown).
    \usepackage{graphicx}
    % We will generate all images so they have a width \maxwidth. This means
    % that they will get their normal width if they fit onto the page, but
    % are scaled down if they would overflow the margins.
    \makeatletter
    \def\maxwidth{\ifdim\Gin@nat@width>\linewidth\linewidth
    \else\Gin@nat@width\fi}
    \makeatother
    \let\Oldincludegraphics\includegraphics
    % Set max figure width to be 80% of text width, for now hardcoded.
    \renewcommand{\includegraphics}[1]{\Oldincludegraphics[width=.8\maxwidth]{#1}}
    % Ensure that by default, figures have no caption (until we provide a
    % proper Figure object with a Caption API and a way to capture that
    % in the conversion process - todo).
    \usepackage{caption}
    \DeclareCaptionLabelFormat{nolabel}{}
    \captionsetup{labelformat=nolabel}

    \usepackage{adjustbox} % Used to constrain images to a maximum size 
    \usepackage{xcolor} % Allow colors to be defined
    \usepackage{enumerate} % Needed for markdown enumerations to work
    \usepackage{geometry} % Used to adjust the document margins
    \usepackage{amsmath} % Equations
    \usepackage{amssymb} % Equations
    \usepackage{textcomp} % defines textquotesingle
    % Hack from http://tex.stackexchange.com/a/47451/13684:
    \AtBeginDocument{%
        \def\PYZsq{\textquotesingle}% Upright quotes in Pygmentized code
    }
    \usepackage{upquote} % Upright quotes for verbatim code
    \usepackage{eurosym} % defines \euro
    \usepackage[mathletters]{ucs} % Extended unicode (utf-8) support
    \usepackage[utf8x]{inputenc} % Allow utf-8 characters in the tex document
    \usepackage{fancyvrb} % verbatim replacement that allows latex
    \usepackage{grffile} % extends the file name processing of package graphics 
                         % to support a larger range 
    % The hyperref package gives us a pdf with properly built
    % internal navigation ('pdf bookmarks' for the table of contents,
    % internal cross-reference links, web links for URLs, etc.)
    \usepackage{hyperref}
    \usepackage{longtable} % longtable support required by pandoc >1.10
    \usepackage{booktabs}  % table support for pandoc > 1.12.2
    \usepackage[inline]{enumitem} % IRkernel/repr support (it uses the enumerate* environment)
    \usepackage[normalem]{ulem} % ulem is needed to support strikethroughs (\sout)
                                % normalem makes italics be italics, not underlines
    

    
    
    % Colors for the hyperref package
    \definecolor{urlcolor}{rgb}{0,.145,.698}
    \definecolor{linkcolor}{rgb}{.71,0.21,0.01}
    \definecolor{citecolor}{rgb}{.12,.54,.11}

    % ANSI colors
    \definecolor{ansi-black}{HTML}{3E424D}
    \definecolor{ansi-black-intense}{HTML}{282C36}
    \definecolor{ansi-red}{HTML}{E75C58}
    \definecolor{ansi-red-intense}{HTML}{B22B31}
    \definecolor{ansi-green}{HTML}{00A250}
    \definecolor{ansi-green-intense}{HTML}{007427}
    \definecolor{ansi-yellow}{HTML}{DDB62B}
    \definecolor{ansi-yellow-intense}{HTML}{B27D12}
    \definecolor{ansi-blue}{HTML}{208FFB}
    \definecolor{ansi-blue-intense}{HTML}{0065CA}
    \definecolor{ansi-magenta}{HTML}{D160C4}
    \definecolor{ansi-magenta-intense}{HTML}{A03196}
    \definecolor{ansi-cyan}{HTML}{60C6C8}
    \definecolor{ansi-cyan-intense}{HTML}{258F8F}
    \definecolor{ansi-white}{HTML}{C5C1B4}
    \definecolor{ansi-white-intense}{HTML}{A1A6B2}

    % commands and environments needed by pandoc snippets
    % extracted from the output of `pandoc -s`
    \providecommand{\tightlist}{%
      \setlength{\itemsep}{0pt}\setlength{\parskip}{0pt}}
    \DefineVerbatimEnvironment{Highlighting}{Verbatim}{commandchars=\\\{\}}
    % Add ',fontsize=\small' for more characters per line
    \newenvironment{Shaded}{}{}
    \newcommand{\KeywordTok}[1]{\textcolor[rgb]{0.00,0.44,0.13}{\textbf{{#1}}}}
    \newcommand{\DataTypeTok}[1]{\textcolor[rgb]{0.56,0.13,0.00}{{#1}}}
    \newcommand{\DecValTok}[1]{\textcolor[rgb]{0.25,0.63,0.44}{{#1}}}
    \newcommand{\BaseNTok}[1]{\textcolor[rgb]{0.25,0.63,0.44}{{#1}}}
    \newcommand{\FloatTok}[1]{\textcolor[rgb]{0.25,0.63,0.44}{{#1}}}
    \newcommand{\CharTok}[1]{\textcolor[rgb]{0.25,0.44,0.63}{{#1}}}
    \newcommand{\StringTok}[1]{\textcolor[rgb]{0.25,0.44,0.63}{{#1}}}
    \newcommand{\CommentTok}[1]{\textcolor[rgb]{0.38,0.63,0.69}{\textit{{#1}}}}
    \newcommand{\OtherTok}[1]{\textcolor[rgb]{0.00,0.44,0.13}{{#1}}}
    \newcommand{\AlertTok}[1]{\textcolor[rgb]{1.00,0.00,0.00}{\textbf{{#1}}}}
    \newcommand{\FunctionTok}[1]{\textcolor[rgb]{0.02,0.16,0.49}{{#1}}}
    \newcommand{\RegionMarkerTok}[1]{{#1}}
    \newcommand{\ErrorTok}[1]{\textcolor[rgb]{1.00,0.00,0.00}{\textbf{{#1}}}}
    \newcommand{\NormalTok}[1]{{#1}}
    
    % Additional commands for more recent versions of Pandoc
    \newcommand{\ConstantTok}[1]{\textcolor[rgb]{0.53,0.00,0.00}{{#1}}}
    \newcommand{\SpecialCharTok}[1]{\textcolor[rgb]{0.25,0.44,0.63}{{#1}}}
    \newcommand{\VerbatimStringTok}[1]{\textcolor[rgb]{0.25,0.44,0.63}{{#1}}}
    \newcommand{\SpecialStringTok}[1]{\textcolor[rgb]{0.73,0.40,0.53}{{#1}}}
    \newcommand{\ImportTok}[1]{{#1}}
    \newcommand{\DocumentationTok}[1]{\textcolor[rgb]{0.73,0.13,0.13}{\textit{{#1}}}}
    \newcommand{\AnnotationTok}[1]{\textcolor[rgb]{0.38,0.63,0.69}{\textbf{\textit{{#1}}}}}
    \newcommand{\CommentVarTok}[1]{\textcolor[rgb]{0.38,0.63,0.69}{\textbf{\textit{{#1}}}}}
    \newcommand{\VariableTok}[1]{\textcolor[rgb]{0.10,0.09,0.49}{{#1}}}
    \newcommand{\ControlFlowTok}[1]{\textcolor[rgb]{0.00,0.44,0.13}{\textbf{{#1}}}}
    \newcommand{\OperatorTok}[1]{\textcolor[rgb]{0.40,0.40,0.40}{{#1}}}
    \newcommand{\BuiltInTok}[1]{{#1}}
    \newcommand{\ExtensionTok}[1]{{#1}}
    \newcommand{\PreprocessorTok}[1]{\textcolor[rgb]{0.74,0.48,0.00}{{#1}}}
    \newcommand{\AttributeTok}[1]{\textcolor[rgb]{0.49,0.56,0.16}{{#1}}}
    \newcommand{\InformationTok}[1]{\textcolor[rgb]{0.38,0.63,0.69}{\textbf{\textit{{#1}}}}}
    \newcommand{\WarningTok}[1]{\textcolor[rgb]{0.38,0.63,0.69}{\textbf{\textit{{#1}}}}}
    
    
    % Define a nice break command that doesn't care if a line doesn't already
    % exist.
    \def\br{\hspace*{\fill} \\* }
    % Math Jax compatability definitions
    \def\gt{>}
    \def\lt{<}
    % Document parameters
    \title{Analysis of the Stroop Effect}
    
    
    

    % Pygments definitions
    
\makeatletter
\def\PY@reset{\let\PY@it=\relax \let\PY@bf=\relax%
    \let\PY@ul=\relax \let\PY@tc=\relax%
    \let\PY@bc=\relax \let\PY@ff=\relax}
\def\PY@tok#1{\csname PY@tok@#1\endcsname}
\def\PY@toks#1+{\ifx\relax#1\empty\else%
    \PY@tok{#1}\expandafter\PY@toks\fi}
\def\PY@do#1{\PY@bc{\PY@tc{\PY@ul{%
    \PY@it{\PY@bf{\PY@ff{#1}}}}}}}
\def\PY#1#2{\PY@reset\PY@toks#1+\relax+\PY@do{#2}}

\expandafter\def\csname PY@tok@w\endcsname{\def\PY@tc##1{\textcolor[rgb]{0.73,0.73,0.73}{##1}}}
\expandafter\def\csname PY@tok@c\endcsname{\let\PY@it=\textit\def\PY@tc##1{\textcolor[rgb]{0.25,0.50,0.50}{##1}}}
\expandafter\def\csname PY@tok@cp\endcsname{\def\PY@tc##1{\textcolor[rgb]{0.74,0.48,0.00}{##1}}}
\expandafter\def\csname PY@tok@k\endcsname{\let\PY@bf=\textbf\def\PY@tc##1{\textcolor[rgb]{0.00,0.50,0.00}{##1}}}
\expandafter\def\csname PY@tok@kp\endcsname{\def\PY@tc##1{\textcolor[rgb]{0.00,0.50,0.00}{##1}}}
\expandafter\def\csname PY@tok@kt\endcsname{\def\PY@tc##1{\textcolor[rgb]{0.69,0.00,0.25}{##1}}}
\expandafter\def\csname PY@tok@o\endcsname{\def\PY@tc##1{\textcolor[rgb]{0.40,0.40,0.40}{##1}}}
\expandafter\def\csname PY@tok@ow\endcsname{\let\PY@bf=\textbf\def\PY@tc##1{\textcolor[rgb]{0.67,0.13,1.00}{##1}}}
\expandafter\def\csname PY@tok@nb\endcsname{\def\PY@tc##1{\textcolor[rgb]{0.00,0.50,0.00}{##1}}}
\expandafter\def\csname PY@tok@nf\endcsname{\def\PY@tc##1{\textcolor[rgb]{0.00,0.00,1.00}{##1}}}
\expandafter\def\csname PY@tok@nc\endcsname{\let\PY@bf=\textbf\def\PY@tc##1{\textcolor[rgb]{0.00,0.00,1.00}{##1}}}
\expandafter\def\csname PY@tok@nn\endcsname{\let\PY@bf=\textbf\def\PY@tc##1{\textcolor[rgb]{0.00,0.00,1.00}{##1}}}
\expandafter\def\csname PY@tok@ne\endcsname{\let\PY@bf=\textbf\def\PY@tc##1{\textcolor[rgb]{0.82,0.25,0.23}{##1}}}
\expandafter\def\csname PY@tok@nv\endcsname{\def\PY@tc##1{\textcolor[rgb]{0.10,0.09,0.49}{##1}}}
\expandafter\def\csname PY@tok@no\endcsname{\def\PY@tc##1{\textcolor[rgb]{0.53,0.00,0.00}{##1}}}
\expandafter\def\csname PY@tok@nl\endcsname{\def\PY@tc##1{\textcolor[rgb]{0.63,0.63,0.00}{##1}}}
\expandafter\def\csname PY@tok@ni\endcsname{\let\PY@bf=\textbf\def\PY@tc##1{\textcolor[rgb]{0.60,0.60,0.60}{##1}}}
\expandafter\def\csname PY@tok@na\endcsname{\def\PY@tc##1{\textcolor[rgb]{0.49,0.56,0.16}{##1}}}
\expandafter\def\csname PY@tok@nt\endcsname{\let\PY@bf=\textbf\def\PY@tc##1{\textcolor[rgb]{0.00,0.50,0.00}{##1}}}
\expandafter\def\csname PY@tok@nd\endcsname{\def\PY@tc##1{\textcolor[rgb]{0.67,0.13,1.00}{##1}}}
\expandafter\def\csname PY@tok@s\endcsname{\def\PY@tc##1{\textcolor[rgb]{0.73,0.13,0.13}{##1}}}
\expandafter\def\csname PY@tok@sd\endcsname{\let\PY@it=\textit\def\PY@tc##1{\textcolor[rgb]{0.73,0.13,0.13}{##1}}}
\expandafter\def\csname PY@tok@si\endcsname{\let\PY@bf=\textbf\def\PY@tc##1{\textcolor[rgb]{0.73,0.40,0.53}{##1}}}
\expandafter\def\csname PY@tok@se\endcsname{\let\PY@bf=\textbf\def\PY@tc##1{\textcolor[rgb]{0.73,0.40,0.13}{##1}}}
\expandafter\def\csname PY@tok@sr\endcsname{\def\PY@tc##1{\textcolor[rgb]{0.73,0.40,0.53}{##1}}}
\expandafter\def\csname PY@tok@ss\endcsname{\def\PY@tc##1{\textcolor[rgb]{0.10,0.09,0.49}{##1}}}
\expandafter\def\csname PY@tok@sx\endcsname{\def\PY@tc##1{\textcolor[rgb]{0.00,0.50,0.00}{##1}}}
\expandafter\def\csname PY@tok@m\endcsname{\def\PY@tc##1{\textcolor[rgb]{0.40,0.40,0.40}{##1}}}
\expandafter\def\csname PY@tok@gh\endcsname{\let\PY@bf=\textbf\def\PY@tc##1{\textcolor[rgb]{0.00,0.00,0.50}{##1}}}
\expandafter\def\csname PY@tok@gu\endcsname{\let\PY@bf=\textbf\def\PY@tc##1{\textcolor[rgb]{0.50,0.00,0.50}{##1}}}
\expandafter\def\csname PY@tok@gd\endcsname{\def\PY@tc##1{\textcolor[rgb]{0.63,0.00,0.00}{##1}}}
\expandafter\def\csname PY@tok@gi\endcsname{\def\PY@tc##1{\textcolor[rgb]{0.00,0.63,0.00}{##1}}}
\expandafter\def\csname PY@tok@gr\endcsname{\def\PY@tc##1{\textcolor[rgb]{1.00,0.00,0.00}{##1}}}
\expandafter\def\csname PY@tok@ge\endcsname{\let\PY@it=\textit}
\expandafter\def\csname PY@tok@gs\endcsname{\let\PY@bf=\textbf}
\expandafter\def\csname PY@tok@gp\endcsname{\let\PY@bf=\textbf\def\PY@tc##1{\textcolor[rgb]{0.00,0.00,0.50}{##1}}}
\expandafter\def\csname PY@tok@go\endcsname{\def\PY@tc##1{\textcolor[rgb]{0.53,0.53,0.53}{##1}}}
\expandafter\def\csname PY@tok@gt\endcsname{\def\PY@tc##1{\textcolor[rgb]{0.00,0.27,0.87}{##1}}}
\expandafter\def\csname PY@tok@err\endcsname{\def\PY@bc##1{\setlength{\fboxsep}{0pt}\fcolorbox[rgb]{1.00,0.00,0.00}{1,1,1}{\strut ##1}}}
\expandafter\def\csname PY@tok@kc\endcsname{\let\PY@bf=\textbf\def\PY@tc##1{\textcolor[rgb]{0.00,0.50,0.00}{##1}}}
\expandafter\def\csname PY@tok@kd\endcsname{\let\PY@bf=\textbf\def\PY@tc##1{\textcolor[rgb]{0.00,0.50,0.00}{##1}}}
\expandafter\def\csname PY@tok@kn\endcsname{\let\PY@bf=\textbf\def\PY@tc##1{\textcolor[rgb]{0.00,0.50,0.00}{##1}}}
\expandafter\def\csname PY@tok@kr\endcsname{\let\PY@bf=\textbf\def\PY@tc##1{\textcolor[rgb]{0.00,0.50,0.00}{##1}}}
\expandafter\def\csname PY@tok@bp\endcsname{\def\PY@tc##1{\textcolor[rgb]{0.00,0.50,0.00}{##1}}}
\expandafter\def\csname PY@tok@fm\endcsname{\def\PY@tc##1{\textcolor[rgb]{0.00,0.00,1.00}{##1}}}
\expandafter\def\csname PY@tok@vc\endcsname{\def\PY@tc##1{\textcolor[rgb]{0.10,0.09,0.49}{##1}}}
\expandafter\def\csname PY@tok@vg\endcsname{\def\PY@tc##1{\textcolor[rgb]{0.10,0.09,0.49}{##1}}}
\expandafter\def\csname PY@tok@vi\endcsname{\def\PY@tc##1{\textcolor[rgb]{0.10,0.09,0.49}{##1}}}
\expandafter\def\csname PY@tok@vm\endcsname{\def\PY@tc##1{\textcolor[rgb]{0.10,0.09,0.49}{##1}}}
\expandafter\def\csname PY@tok@sa\endcsname{\def\PY@tc##1{\textcolor[rgb]{0.73,0.13,0.13}{##1}}}
\expandafter\def\csname PY@tok@sb\endcsname{\def\PY@tc##1{\textcolor[rgb]{0.73,0.13,0.13}{##1}}}
\expandafter\def\csname PY@tok@sc\endcsname{\def\PY@tc##1{\textcolor[rgb]{0.73,0.13,0.13}{##1}}}
\expandafter\def\csname PY@tok@dl\endcsname{\def\PY@tc##1{\textcolor[rgb]{0.73,0.13,0.13}{##1}}}
\expandafter\def\csname PY@tok@s2\endcsname{\def\PY@tc##1{\textcolor[rgb]{0.73,0.13,0.13}{##1}}}
\expandafter\def\csname PY@tok@sh\endcsname{\def\PY@tc##1{\textcolor[rgb]{0.73,0.13,0.13}{##1}}}
\expandafter\def\csname PY@tok@s1\endcsname{\def\PY@tc##1{\textcolor[rgb]{0.73,0.13,0.13}{##1}}}
\expandafter\def\csname PY@tok@mb\endcsname{\def\PY@tc##1{\textcolor[rgb]{0.40,0.40,0.40}{##1}}}
\expandafter\def\csname PY@tok@mf\endcsname{\def\PY@tc##1{\textcolor[rgb]{0.40,0.40,0.40}{##1}}}
\expandafter\def\csname PY@tok@mh\endcsname{\def\PY@tc##1{\textcolor[rgb]{0.40,0.40,0.40}{##1}}}
\expandafter\def\csname PY@tok@mi\endcsname{\def\PY@tc##1{\textcolor[rgb]{0.40,0.40,0.40}{##1}}}
\expandafter\def\csname PY@tok@il\endcsname{\def\PY@tc##1{\textcolor[rgb]{0.40,0.40,0.40}{##1}}}
\expandafter\def\csname PY@tok@mo\endcsname{\def\PY@tc##1{\textcolor[rgb]{0.40,0.40,0.40}{##1}}}
\expandafter\def\csname PY@tok@ch\endcsname{\let\PY@it=\textit\def\PY@tc##1{\textcolor[rgb]{0.25,0.50,0.50}{##1}}}
\expandafter\def\csname PY@tok@cm\endcsname{\let\PY@it=\textit\def\PY@tc##1{\textcolor[rgb]{0.25,0.50,0.50}{##1}}}
\expandafter\def\csname PY@tok@cpf\endcsname{\let\PY@it=\textit\def\PY@tc##1{\textcolor[rgb]{0.25,0.50,0.50}{##1}}}
\expandafter\def\csname PY@tok@c1\endcsname{\let\PY@it=\textit\def\PY@tc##1{\textcolor[rgb]{0.25,0.50,0.50}{##1}}}
\expandafter\def\csname PY@tok@cs\endcsname{\let\PY@it=\textit\def\PY@tc##1{\textcolor[rgb]{0.25,0.50,0.50}{##1}}}

\def\PYZbs{\char`\\}
\def\PYZus{\char`\_}
\def\PYZob{\char`\{}
\def\PYZcb{\char`\}}
\def\PYZca{\char`\^}
\def\PYZam{\char`\&}
\def\PYZlt{\char`\<}
\def\PYZgt{\char`\>}
\def\PYZsh{\char`\#}
\def\PYZpc{\char`\%}
\def\PYZdl{\char`\$}
\def\PYZhy{\char`\-}
\def\PYZsq{\char`\'}
\def\PYZdq{\char`\"}
\def\PYZti{\char`\~}
% for compatibility with earlier versions
\def\PYZat{@}
\def\PYZlb{[}
\def\PYZrb{]}
\makeatother


    % Exact colors from NB
    \definecolor{incolor}{rgb}{0.0, 0.0, 0.5}
    \definecolor{outcolor}{rgb}{0.545, 0.0, 0.0}



    
    % Prevent overflowing lines due to hard-to-break entities
    \sloppy 
    % Setup hyperref package
    \hypersetup{
      breaklinks=true,  % so long urls are correctly broken across lines
      colorlinks=true,
      urlcolor=urlcolor,
      linkcolor=linkcolor,
      citecolor=citecolor,
      }
    % Slightly bigger margins than the latex defaults
    
    \geometry{verbose,tmargin=1in,bmargin=1in,lmargin=1in,rmargin=1in}
    
    

    \begin{document}
    
    
    \maketitle
    
    

    
    \section{Analysis of the Stroop
Effect}\label{analysis-of-the-stroop-effect}

    \subsection{Introduction}\label{introduction}

    The Stroop Effect is the demonstration of inference in the reaction time
of a task.

The test revoles around an individual calling out the color of ink a
word is written in.

The test is administered in two phases:

\begin{verbatim}
The first phase involves congruent words and colors, in other words the word BLUE is written in BLUE ink.
\end{verbatim}

\begin{figure}
\centering
\includegraphics{attachment:stroop_congruent.png}
\caption{stroop\_congruent.png}
\end{figure}

\begin{verbatim}
The second phase involves incongruent words and colors, or better said, the word BLUE is written in RED ink.
\end{verbatim}

\begin{figure}
\centering
\includegraphics{attachment:stroop_incongruent.png}
\caption{stroop\_incongruent.png}
\end{figure}

The length of time is measured in how long it takes the individual to
call out the correct color of ink in equally sized lists.

    \subsection{Investigation}\label{investigation}

    \begin{itemize}
\item
  \textbf{Dependant Variable:} The dependant variable in this study is
  the response time of the individual in seconds for each of the word
  sets (congruent and incongruent)
\item
  \textbf{Independant Variable:} The independant variable in this study
  is the word set (congruent and incongruent)
\end{itemize}

    \subsubsection{Testing Method}\label{testing-method}

A Dependant Sample t-Test should be used for this study. We will use a
two-tailed test using a sginifigance level of 0.05. A Dependant Sample
t\_Test compares the means of two related groups and determines if there
is a statistically significant difference between them. This type of
test relies on one dependant variable measured on an interval (how long
it takes to correctly idnetify the color of ink the word is written in)
and one categorical variable that has two related groups (the
individuals participating in the study). The individuals participating
in the study are dependant as they are the same group of people
performing the task on both the congruent and incongruent word sets.

    \subsubsection{Hypothesis}\label{hypothesis}

There are two hypothesis for this study.

\textbf{Null Hypothesis} - There is no difference in the time (in
seconds) it takes an individual to correctly identify the color of ink
between the congruent and incongruent word sets. This can be expressed
as:

\subsubsection{\texorpdfstring{\textbf{H\(_{0}\): \(\mu\)\(_{i}\) =
\(\mu\)\(_{c}\)}}{H\_\{0\}: \textbackslash{}mu\_\{i\} = \textbackslash{}mu\_\{c\}}}\label{h_0-mu_i-mu_c}

\textbf{Alternate Hypothesis} - The difference in time (in seconds) it
takes for an individual to correctlly identify the color of ink is
significant between the congruent and incongruent word sets. This
difference could be greater than, lesser than, or just different. This
can be expressed as:

\subsubsection{\texorpdfstring{\textbf{H\(_{A}\): \(\mu\)\(_{i}\)
\textless{} \(\mu\)\(_{c}\) , \(\mu\)\(_{i}\) \textgreater{}
\(\mu\)\(_{c}\) , \(\mu\)\(_{i}\) \(\neq\)
\(\mu\)\(_{c}\)}}{H\_\{A\}: \textbackslash{}mu\_\{i\} \textless{} \textbackslash{}mu\_\{c\} , \textbackslash{}mu\_\{i\} \textgreater{} \textbackslash{}mu\_\{c\} , \textbackslash{}mu\_\{i\} \textbackslash{}neq \textbackslash{}mu\_\{c\}}}\label{h_a-mu_i-mu_c-mu_i-mu_c-mu_i-neq-mu_c}

    \subsection{Analysis}\label{analysis}

    Let's import the necessary Python libraries to perform our analysis

    \begin{Verbatim}[commandchars=\\\{\}]
{\color{incolor}In [{\color{incolor}1}]:} \PY{k+kn}{import} \PY{n+nn}{csv}
        \PY{k+kn}{from} \PY{n+nn}{datetime} \PY{k}{import} \PY{n}{datetime}
        \PY{k+kn}{import} \PY{n+nn}{numpy} \PY{k}{as} \PY{n+nn}{np}
        \PY{k+kn}{import} \PY{n+nn}{pandas} \PY{k}{as} \PY{n+nn}{pd}
        \PY{k+kn}{import} \PY{n+nn}{matplotlib}\PY{n+nn}{.}\PY{n+nn}{pyplot} \PY{k}{as} \PY{n+nn}{plt}
        \PY{k+kn}{from} \PY{n+nn}{math} \PY{k}{import} \PY{o}{*}
        \PY{o}{\PYZpc{}}\PY{k}{matplotlib} inline
\end{Verbatim}


    \begin{Verbatim}[commandchars=\\\{\}]
{\color{incolor}In [{\color{incolor}2}]:} \PY{c+c1}{\PYZsh{} set matplotlib figure size to something easier to look at}
        \PY{n}{plt}\PY{o}{.}\PY{n}{rcParams}\PY{p}{[}\PY{l+s+s1}{\PYZsq{}}\PY{l+s+s1}{figure.dpi}\PY{l+s+s1}{\PYZsq{}}\PY{p}{]} \PY{o}{=} \PY{l+m+mi}{100}
\end{Verbatim}


    We will start by createing a dataframe containing our Stroop data.

    \begin{Verbatim}[commandchars=\\\{\}]
{\color{incolor}In [{\color{incolor}3}]:} \PY{c+c1}{\PYZsh{} create pandas dataframe from our csv dataset}
        \PY{n}{stroop\PYZus{}data} \PY{o}{=} \PY{n}{pd}\PY{o}{.}\PY{n}{read\PYZus{}csv}\PY{p}{(}\PY{l+s+s1}{\PYZsq{}}\PY{l+s+s1}{stroopdata.csv}\PY{l+s+s1}{\PYZsq{}}\PY{p}{)}
\end{Verbatim}


    \begin{Verbatim}[commandchars=\\\{\}]
{\color{incolor}In [{\color{incolor}4}]:} \PY{c+c1}{\PYZsh{} view the first few rows of the dataframe}
        \PY{n}{stroop\PYZus{}data}\PY{o}{.}\PY{n}{head}\PY{p}{(}\PY{p}{)}
\end{Verbatim}


\begin{Verbatim}[commandchars=\\\{\}]
{\color{outcolor}Out[{\color{outcolor}4}]:}    Congruent  Incongruent
        0     12.079       19.278
        1     16.791       18.741
        2      9.564       21.214
        3      8.630       15.687
        4     14.669       22.803
\end{Verbatim}
            
    Since our index starts at zero, we will create a participants column and
assign a value of index + 1 to it.

    Now lets have a look at a statistical view of our data.

    \begin{Verbatim}[commandchars=\\\{\}]
{\color{incolor}In [{\color{incolor}5}]:} \PY{n}{stroop\PYZus{}data}\PY{o}{.}\PY{n}{describe}\PY{p}{(}\PY{p}{)}
\end{Verbatim}


\begin{Verbatim}[commandchars=\\\{\}]
{\color{outcolor}Out[{\color{outcolor}5}]:}        Congruent  Incongruent
        count  24.000000    24.000000
        mean   14.051125    22.015917
        std     3.559358     4.797057
        min     8.630000    15.687000
        25\%    11.895250    18.716750
        50\%    14.356500    21.017500
        75\%    16.200750    24.051500
        max    22.328000    35.255000
\end{Verbatim}
            
    Now we will have a look at this data represented as a Boxplot for an
easier visual analysis.

    \begin{Verbatim}[commandchars=\\\{\}]
{\color{incolor}In [{\color{incolor}6}]:} \PY{c+c1}{\PYZsh{} create boxplot from dataframe}
        \PY{n}{plt}\PY{o}{.}\PY{n}{title}\PY{p}{(}\PY{l+s+s1}{\PYZsq{}}\PY{l+s+s1}{Boxplot Congruent vs Incongruent Word Sets}\PY{l+s+s1}{\PYZsq{}}\PY{p}{)}
        \PY{n}{plt}\PY{o}{.}\PY{n}{ylabel}\PY{p}{(}\PY{l+s+s1}{\PYZsq{}}\PY{l+s+s1}{Time in Seconds}\PY{l+s+s1}{\PYZsq{}}\PY{p}{)}
        
        \PY{n}{stroop\PYZus{}data}\PY{o}{.}\PY{n}{boxplot}\PY{p}{(}\PY{p}{[}\PY{l+s+s1}{\PYZsq{}}\PY{l+s+s1}{Congruent}\PY{l+s+s1}{\PYZsq{}}\PY{p}{,} \PY{l+s+s1}{\PYZsq{}}\PY{l+s+s1}{Incongruent}\PY{l+s+s1}{\PYZsq{}}\PY{p}{]}\PY{p}{)}
        \PY{n}{plt}\PY{o}{.}\PY{n}{show}\PY{p}{(}\PY{p}{)}
\end{Verbatim}


    \begin{center}
    \adjustimage{max size={0.9\linewidth}{0.9\paperheight}}{output_18_0.png}
    \end{center}
    { \hspace*{\fill} \\}
    
    We can clearly see there is a difference between the congruent and
incongruent word sets.

Next we can take a look at the Histograms for each word set.

    \begin{Verbatim}[commandchars=\\\{\}]
{\color{incolor}In [{\color{incolor}7}]:} \PY{c+c1}{\PYZsh{} function to create histograms from dataframe, column}
        \PY{k}{def} \PY{n+nf}{create\PYZus{}hist}\PY{p}{(}\PY{n}{df}\PY{p}{,} \PY{n}{col}\PY{p}{)}\PY{p}{:}
            \PY{n}{df}\PY{o}{.}\PY{n}{hist}\PY{p}{(}\PY{n}{column} \PY{o}{=} \PY{n}{col}\PY{p}{,} \PY{n}{bins} \PY{o}{=} \PY{p}{[}\PY{l+m+mi}{5}\PY{p}{,}\PY{l+m+mi}{10}\PY{p}{,}\PY{l+m+mi}{15}\PY{p}{,}\PY{l+m+mi}{20}\PY{p}{,}\PY{l+m+mi}{25}\PY{p}{,}\PY{l+m+mi}{30}\PY{p}{,}\PY{l+m+mi}{35}\PY{p}{,}\PY{l+m+mi}{40}\PY{p}{]}\PY{p}{)}
            \PY{n}{plt}\PY{o}{.}\PY{n}{ylabel}\PY{p}{(}\PY{l+s+s1}{\PYZsq{}}\PY{l+s+s1}{Number of Participants}\PY{l+s+s1}{\PYZsq{}}\PY{p}{)}
            \PY{n}{plt}\PY{o}{.}\PY{n}{xlabel}\PY{p}{(}\PY{l+s+s1}{\PYZsq{}}\PY{l+s+s1}{Time in Seconds}\PY{l+s+s1}{\PYZsq{}}\PY{p}{)}
            \PY{n}{plt}\PY{o}{.}\PY{n}{show}\PY{p}{(}\PY{p}{)}
        
        \PY{c+c1}{\PYZsh{} pass dataframe, column to the function}
        \PY{n}{create\PYZus{}hist}\PY{p}{(}\PY{n}{stroop\PYZus{}data}\PY{p}{,} \PY{l+s+s1}{\PYZsq{}}\PY{l+s+s1}{Congruent}\PY{l+s+s1}{\PYZsq{}}\PY{p}{)}
        \PY{n}{create\PYZus{}hist}\PY{p}{(}\PY{n}{stroop\PYZus{}data}\PY{p}{,} \PY{l+s+s1}{\PYZsq{}}\PY{l+s+s1}{Incongruent}\PY{l+s+s1}{\PYZsq{}}\PY{p}{)}
\end{Verbatim}


    \begin{center}
    \adjustimage{max size={0.9\linewidth}{0.9\paperheight}}{output_20_0.png}
    \end{center}
    { \hspace*{\fill} \\}
    
    \begin{center}
    \adjustimage{max size={0.9\linewidth}{0.9\paperheight}}{output_20_1.png}
    \end{center}
    { \hspace*{\fill} \\}
    
    From the above historgrams we see that there is definite differnece
between the time (in seconds) when completing the congruent and
incongruent word lists. Participants were able to complete the congruent
word list quicker.

    \subsection{Statistical Test}\label{statistical-test}

    We'll start by calculating the difference between the congruent test and
the incongruent test for each participant and store the value in the
dataframe.

Difference = Incongruent - Congruent

    \begin{Verbatim}[commandchars=\\\{\}]
{\color{incolor}In [{\color{incolor}8}]:} \PY{c+c1}{\PYZsh{} calculate the difference between the Incongruent test and the Congruent test}
        \PY{n}{stroop\PYZus{}data}\PY{p}{[}\PY{l+s+s1}{\PYZsq{}}\PY{l+s+s1}{Diff}\PY{l+s+s1}{\PYZsq{}}\PY{p}{]} \PY{o}{=} \PY{n}{stroop\PYZus{}data}\PY{p}{[}\PY{l+s+s1}{\PYZsq{}}\PY{l+s+s1}{Incongruent}\PY{l+s+s1}{\PYZsq{}}\PY{p}{]} \PY{o}{\PYZhy{}} \PY{n}{stroop\PYZus{}data}\PY{p}{[}\PY{l+s+s1}{\PYZsq{}}\PY{l+s+s1}{Congruent}\PY{l+s+s1}{\PYZsq{}}\PY{p}{]}
        
        \PY{n}{stroop\PYZus{}data}\PY{o}{.}\PY{n}{head}\PY{p}{(}\PY{p}{)}
\end{Verbatim}


\begin{Verbatim}[commandchars=\\\{\}]
{\color{outcolor}Out[{\color{outcolor}8}]:}    Congruent  Incongruent    Diff
        0     12.079       19.278   7.199
        1     16.791       18.741   1.950
        2      9.564       21.214  11.650
        3      8.630       15.687   7.057
        4     14.669       22.803   8.134
\end{Verbatim}
            
    Next we will calculate the deviation for each particpant.

Deviation = Diffirence - Mean of all Differences

    \begin{Verbatim}[commandchars=\\\{\}]
{\color{incolor}In [{\color{incolor}9}]:} \PY{c+c1}{\PYZsh{} calculat the deviation between the differnece and the mean of all differences}
        \PY{n}{stroop\PYZus{}data}\PY{p}{[}\PY{l+s+s1}{\PYZsq{}}\PY{l+s+s1}{Dev}\PY{l+s+s1}{\PYZsq{}}\PY{p}{]} \PY{o}{=} \PY{n}{stroop\PYZus{}data}\PY{p}{[}\PY{l+s+s1}{\PYZsq{}}\PY{l+s+s1}{Diff}\PY{l+s+s1}{\PYZsq{}}\PY{p}{]} \PY{o}{\PYZhy{}} \PY{n}{stroop\PYZus{}data}\PY{p}{[}\PY{l+s+s1}{\PYZsq{}}\PY{l+s+s1}{Diff}\PY{l+s+s1}{\PYZsq{}}\PY{p}{]}\PY{o}{.}\PY{n}{mean}\PY{p}{(}\PY{p}{)}
        
        \PY{n}{stroop\PYZus{}data}\PY{o}{.}\PY{n}{head}\PY{p}{(}\PY{p}{)}
\end{Verbatim}


\begin{Verbatim}[commandchars=\\\{\}]
{\color{outcolor}Out[{\color{outcolor}9}]:}    Congruent  Incongruent    Diff       Dev
        0     12.079       19.278   7.199 -0.765792
        1     16.791       18.741   1.950 -6.014792
        2      9.564       21.214  11.650  3.685208
        3      8.630       15.687   7.057 -0.907792
        4     14.669       22.803   8.134  0.169208
\end{Verbatim}
            
    Then we will calculate the Squared Differences for each participant.

Square Deviation = Deviation * Deviation

    \begin{Verbatim}[commandchars=\\\{\}]
{\color{incolor}In [{\color{incolor}10}]:} \PY{c+c1}{\PYZsh{} caluculate the Squared Deviation}
         \PY{n}{stroop\PYZus{}data}\PY{p}{[}\PY{l+s+s1}{\PYZsq{}}\PY{l+s+s1}{SQD}\PY{l+s+s1}{\PYZsq{}}\PY{p}{]} \PY{o}{=} \PY{n}{stroop\PYZus{}data}\PY{p}{[}\PY{l+s+s1}{\PYZsq{}}\PY{l+s+s1}{Dev}\PY{l+s+s1}{\PYZsq{}}\PY{p}{]} \PY{o}{*} \PY{n}{stroop\PYZus{}data}\PY{p}{[}\PY{l+s+s1}{\PYZsq{}}\PY{l+s+s1}{Dev}\PY{l+s+s1}{\PYZsq{}}\PY{p}{]}
         
         \PY{n}{stroop\PYZus{}data}\PY{o}{.}\PY{n}{head}\PY{p}{(}\PY{p}{)}
\end{Verbatim}


\begin{Verbatim}[commandchars=\\\{\}]
{\color{outcolor}Out[{\color{outcolor}10}]:}    Congruent  Incongruent    Diff       Dev        SQD
         0     12.079       19.278   7.199 -0.765792   0.586437
         1     16.791       18.741   1.950 -6.014792  36.177719
         2      9.564       21.214  11.650  3.685208  13.580760
         3      8.630       15.687   7.057 -0.907792   0.824086
         4     14.669       22.803   8.134  0.169208   0.028631
\end{Verbatim}
            
    Next lets calculate the difference of the Congruent mean and the
Incongruent mean as the Mean.

    \begin{Verbatim}[commandchars=\\\{\}]
{\color{incolor}In [{\color{incolor}11}]:} \PY{c+c1}{\PYZsh{} create variables to hold the mean of Congruent data and Incongruent data respectively}
         \PY{n}{c\PYZus{}mean} \PY{o}{=} \PY{n}{stroop\PYZus{}data}\PY{p}{[}\PY{l+s+s1}{\PYZsq{}}\PY{l+s+s1}{Congruent}\PY{l+s+s1}{\PYZsq{}}\PY{p}{]}\PY{o}{.}\PY{n}{mean}\PY{p}{(}\PY{p}{)}
         \PY{n}{i\PYZus{}mean} \PY{o}{=} \PY{n}{stroop\PYZus{}data}\PY{p}{[}\PY{l+s+s1}{\PYZsq{}}\PY{l+s+s1}{Incongruent}\PY{l+s+s1}{\PYZsq{}}\PY{p}{]}\PY{o}{.}\PY{n}{mean}\PY{p}{(}\PY{p}{)}
         
         \PY{c+c1}{\PYZsh{} calculate the mean}
         \PY{n}{mean} \PY{o}{=} \PY{n}{i\PYZus{}mean} \PY{o}{\PYZhy{}} \PY{n}{c\PYZus{}mean}
         
         \PY{n+nb}{print}\PY{p}{(}\PY{n+nb}{round}\PY{p}{(}\PY{n}{mean}\PY{p}{,} \PY{l+m+mi}{2}\PY{p}{)}\PY{p}{)}
\end{Verbatim}


    \begin{Verbatim}[commandchars=\\\{\}]
7.96

    \end{Verbatim}

    The Mean is expressed as:

\subsubsection{M: 7.96}\label{m-7.96}

    Now we will calculate the Sum of the Squared Differences.

    \begin{Verbatim}[commandchars=\\\{\}]
{\color{incolor}In [{\color{incolor}12}]:} \PY{n}{sum\PYZus{}sqr\PYZus{}diff} \PY{o}{=} \PY{n}{stroop\PYZus{}data}\PY{p}{[}\PY{l+s+s1}{\PYZsq{}}\PY{l+s+s1}{SQD}\PY{l+s+s1}{\PYZsq{}}\PY{p}{]}\PY{o}{.}\PY{n}{sum}\PY{p}{(}\PY{p}{)}
         
         \PY{n+nb}{print}\PY{p}{(}\PY{n+nb}{round}\PY{p}{(}\PY{n}{sum\PYZus{}sqr\PYZus{}diff}\PY{p}{,} \PY{l+m+mi}{2}\PY{p}{)}\PY{p}{)}
\end{Verbatim}


    \begin{Verbatim}[commandchars=\\\{\}]
544.33

    \end{Verbatim}

    The Sum of Squares is expressed as:

\subsubsection{SS: 544.33}\label{ss-544.33}

    Lets calculate our Sample Variance now.

    \begin{Verbatim}[commandchars=\\\{\}]
{\color{incolor}In [{\color{incolor}13}]:} \PY{c+c1}{\PYZsh{} get number of participants}
         \PY{n}{n} \PY{o}{=} \PY{n+nb}{len}\PY{p}{(}\PY{n}{stroop\PYZus{}data}\PY{p}{)}
         
         \PY{n}{sample\PYZus{}var} \PY{o}{=} \PY{n}{sum\PYZus{}sqr\PYZus{}diff} \PY{o}{/} \PY{p}{(}\PY{n}{n} \PY{o}{\PYZhy{}} \PY{l+m+mi}{1}\PY{p}{)}
         
         \PY{n+nb}{print}\PY{p}{(}\PY{n+nb}{round}\PY{p}{(}\PY{n}{sample\PYZus{}var}\PY{p}{,} \PY{l+m+mi}{2}\PY{p}{)}\PY{p}{)}
\end{Verbatim}


    \begin{Verbatim}[commandchars=\\\{\}]
23.67

    \end{Verbatim}

    The Sample Variance is the Sum of Squares divided by the Degrees of
Freedom (sample size minus 1). This is expressed as:

\subsubsection{\texorpdfstring{S\(^{2}\) =
SS/df}{S\^{}\{2\} = SS/df}}\label{s2-ssdf}

\subsubsection{\texorpdfstring{S\(^{2}\) = 544.33 / (24-1) =
23.67}{S\^{}\{2\} = 544.33 / (24-1) = 23.67}}\label{s2-544.33-24-1-23.67}

    Next we can calculate the Mean of the Sample Variance.

    \begin{Verbatim}[commandchars=\\\{\}]
{\color{incolor}In [{\color{incolor}14}]:} \PY{n}{sample\PYZus{}var\PYZus{}mean} \PY{o}{=} \PY{n}{sample\PYZus{}var} \PY{o}{/} \PY{n}{n}
         
         \PY{n+nb}{print}\PY{p}{(}\PY{n+nb}{round}\PY{p}{(}\PY{n}{sample\PYZus{}var\PYZus{}mean}\PY{p}{,} \PY{l+m+mi}{2}\PY{p}{)}\PY{p}{)}
\end{Verbatim}


    \begin{Verbatim}[commandchars=\\\{\}]
0.99

    \end{Verbatim}

    The Mean of the Sample Variance can be calculated by dividing the Sample
Varience by the Sample Size. This is expressed as:

\subsubsection{\texorpdfstring{S\(^{2}\)\(_{M}\) = S\(^{2}\) /
N}{S\^{}\{2\}\_\{M\} = S\^{}\{2\} / N}}\label{s2_m-s2-n}

\subsubsection{\texorpdfstring{S\(^{2}\)\(_{M}\) = 23.67 / 24 =
0.99}{S\^{}\{2\}\_\{M\} = 23.67 / 24 = 0.99}}\label{s2_m-23.67-24-0.99}

    \begin{Verbatim}[commandchars=\\\{\}]
{\color{incolor}In [{\color{incolor}15}]:} \PY{n}{sample\PYZus{}var\PYZus{}mean\PYZus{}std\PYZus{}dev} \PY{o}{=} \PY{n}{sqrt}\PY{p}{(}\PY{n}{sample\PYZus{}var\PYZus{}mean}\PY{p}{)}
         
         \PY{n+nb}{print}\PY{p}{(}\PY{n+nb}{round}\PY{p}{(}\PY{n}{sample\PYZus{}var\PYZus{}mean\PYZus{}std\PYZus{}dev}\PY{p}{,} \PY{l+m+mi}{2}\PY{p}{)}\PY{p}{)}
\end{Verbatim}


    \begin{Verbatim}[commandchars=\\\{\}]
0.99

    \end{Verbatim}

    The Sample Variation Standard Deviation can be calculated by finding the
square root of the Sample Variance Mean. This is expressed as:

\subsubsection{\texorpdfstring{S\(_{M}\) =
\(\sqrt{S^2_M}\)}{S\_\{M\} = \textbackslash{}sqrt\{S\^{}2\_M\}}}\label{s_m-sqrts2_m}

\subsubsection{\texorpdfstring{S\(_{M}\) = \(\sqrt{0.99}\) =
0.99}{S\_\{M\} = \textbackslash{}sqrt\{0.99\} = 0.99}}\label{s_m-sqrt0.99-0.99}

    Finally we can calculate our t value.

    \begin{Verbatim}[commandchars=\\\{\}]
{\color{incolor}In [{\color{incolor}17}]:} \PY{n}{u} \PY{o}{=} \PY{l+m+mi}{0}
         
         \PY{n}{t} \PY{o}{=} \PY{p}{(}\PY{n}{mean} \PY{o}{\PYZhy{}} \PY{n}{u}\PY{p}{)} \PY{o}{/} \PY{n}{sample\PYZus{}var\PYZus{}mean\PYZus{}std\PYZus{}dev}
         
         \PY{n+nb}{print}\PY{p}{(}\PY{n}{t}\PY{p}{)}
\end{Verbatim}


    \begin{Verbatim}[commandchars=\\\{\}]
8.020706944109957

    \end{Verbatim}

    The formula to calculate the t value is expressed as:

\subsubsection{\texorpdfstring{t = (M - \(\mu\)) /
S\(_{M}\)}{t = (M - \textbackslash{}mu) / S\_\{M\}}}\label{t-m---mu-s_m}

\subsubsection{t = (7.96 - 0) / 0.99 = 8.02}\label{t-7.96---0-0.99-8.02}

    \subsection{Summary}\label{summary}

    The value of t is 8.020706944109957. The value of p is \textless{}
0.00001. The result is significant at p \(\leq\) 0.05.

    We reject the Null Hypothesis which states there is no difference in the
time (in seconds) it takes an individual to correctly identify the color
of ink between the congruent and incongruent word sets.

Based on this test it is expected that every individual that were to
participate would show a signiificant differnece between the Congruant
and Incongruent word sets.

The results are as I expected them to be and are inline with my own
experience while perfroming the task.


    % Add a bibliography block to the postdoc
    
    
    
    \end{document}
